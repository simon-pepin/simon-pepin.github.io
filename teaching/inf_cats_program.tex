\documentclass{amsart}
\usepackage{amscd}
\usepackage{color}
\usepackage{fontspec}
\usepackage{amsmath}
\usepackage{amsfonts}
\usepackage{amsthm}
\usepackage{amssymb}
\usepackage{bbm}
\usepackage{graphicx}
\usepackage{epstopdf}
\newcommand\hmmax{0}
\newcommand\bmmax{0}
\usepackage{bm}
\usepackage[all,cmtip]{xy}
\usepackage{csquotes}
\usepackage{hyperref}
\usepackage{enumerate}
\usepackage{enumitem}
\usepackage{mathrsfs}
\usepackage{vmargin}
\usepackage{verbatim}
\usepackage{cleveref}
\usepackage{stackrel}
\usepackage{epigraph}
\usepackage{chngcntr}
\usepackage{etoolbox}
\usepackage[backend=biber, doi=false,isbn=false, url=false]{biblatex}
\newtheorem{theo}{Theorem}[section]
\newtheorem{cor}[theo]{Corollary}
\newtheorem{prop}[theo]{Proposition}
\newtheorem{lemma}[theo]{Lemma}
\newtheorem{claim}[theo]{Claim}
\newtheorem{conj}[theo]{Conjecture}
\newtheorem{question}[theo]{Question}
\theoremstyle{definition}
\newtheorem{defi}[theo]{Definition}
\theoremstyle{remark}
\newtheorem{remark}[theo]{Remark}
\newtheorem{notation}[theo]{Notation}
\newcommand{\BA}{{\mathbb{A}}}
\newcommand{\BB}{{\mathbb{B}}}
\newcommand{\BC}{{\mathbb{C}}}
\newcommand{\BD}{{\mathbb{D}}}
\newcommand{\BE}{{\mathbb{E}}}
\newcommand{\BF}{{\mathbb{F}}}
\newcommand{\BG}{{\mathbb{G}}}
\newcommand{\BH}{{\mathbb{H}}}
\newcommand{\BI}{{\mathbb{I}}}
\newcommand{\BJ}{{\mathbb{J}}}
\newcommand{\BK}{{\mathbb{K}}}
\newcommand{\BL}{{\mathbb{L}}}
\newcommand{\BM}{{\mathbb{M}}}
\newcommand{\BN}{{\mathbb{N}}}
\newcommand{\BO}{{\mathbb{O}}}
\newcommand{\BP}{{\mathbb{P}}}
\newcommand{\BQ}{{\mathbb{Q}}}
\newcommand{\BR}{{\mathbb{R}}}
\newcommand{\BS}{{\mathbb{S}}}
\newcommand{\BT}{{\mathbb{T}}}
\newcommand{\BU}{{\mathbb{U}}}
\newcommand{\BV}{{\mathbb{V}}}
\newcommand{\BW}{{\mathbb{W}}}
\newcommand{\BX}{{\mathbb{X}}}
\newcommand{\BY}{{\mathbb{Y}}}
\newcommand{\BZ}{{\mathbb{Z}}}
\newcommand{\Fa}{{\mathfrak{a}}}
\newcommand{\Fb}{{\mathfrak{b}}}
\newcommand{\Fc}{{\mathfrak{c}}}
\newcommand{\Fd}{{\mathfrak{d}}}
\newcommand{\Fe}{{\mathfrak{e}}}
\newcommand{\Ff}{{\mathfrak{f}}}
\newcommand{\Fg}{{\mathfrak{g}}}
\newcommand{\Fh}{{\mathfrak{h}}}
\newcommand{\Fi}{{\mathfrak{i}}}
\newcommand{\Fj}{{\mathfrak{j}}}
\newcommand{\Fk}{{\mathfrak{k}}}
\newcommand{\Fl}{{\mathfrak{l}}}
\newcommand{\Fm}{{\mathfrak{m}}}
\newcommand{\Fn}{{\mathfrak{n}}}
\newcommand{\Fo}{{\mathfrak{o}}}
\newcommand{\Fp}{{\mathfrak{p}}}
\newcommand{\Fq}{{\mathfrak{q}}}
\newcommand{\Fr}{{\mathfrak{r}}}
\newcommand{\Fs}{{\mathfrak{s}}}
\newcommand{\Ft}{{\mathfrak{t}}}
\newcommand{\Fu}{{\mathfrak{u}}}
\newcommand{\Fv}{{\mathfrak{v}}}
\newcommand{\Fw}{{\mathfrak{w}}}
\newcommand{\Fx}{{\mathfrak{x}}}
\newcommand{\Fy}{{\mathfrak{y}}}
\newcommand{\Fz}{{\mathfrak{z}}}
\newcommand{\FA}{{\mathfrak{A}}}
\newcommand{\FB}{{\mathfrak{B}}}
\newcommand{\FC}{{\mathfrak{C}}}
\newcommand{\FD}{{\mathfrak{D}}}
\newcommand{\FE}{{\mathfrak{E}}}
\newcommand{\FF}{{\mathfrak{F}}}
\newcommand{\FG}{{\mathfrak{G}}}
\newcommand{\FH}{{\mathfrak{H}}}
\newcommand{\FI}{{\mathfrak{I}}}
\newcommand{\FJ}{{\mathfrak{J}}}
\newcommand{\FK}{{\mathfrak{K}}}
\newcommand{\FL}{{\mathfrak{L}}}
\newcommand{\FM}{{\mathfrak{M}}}
\newcommand{\FN}{{\mathfrak{N}}}
\newcommand{\FO}{{\mathfrak{O}}}
\newcommand{\FP}{{\mathfrak{P}}}
\newcommand{\FQ}{{\mathfrak{Q}}}
\newcommand{\FR}{{\mathfrak{R}}}
\newcommand{\FS}{{\mathfrak{S}}}
\newcommand{\FT}{{\mathfrak{T}}}
\newcommand{\FU}{{\mathfrak{U}}}
\newcommand{\FV}{{\mathfrak{V}}}
\newcommand{\FW}{{\mathfrak{W}}}
\newcommand{\FX}{{\mathfrak{X}}}
\newcommand{\FY}{{\mathfrak{Y}}}
\newcommand{\FZ}{{\mathfrak{Z}}}
\newcommand{\Ca}{{\mathcal{a}}}
\newcommand{\Cb}{{\mathcal{b}}}
\newcommand{\Cc}{{\mathcal{c}}}
\newcommand{\Cd}{{\mathcal{d}}}
\newcommand{\Ce}{{\mathcal{e}}}
\newcommand{\Cf}{{\mathcal{f}}}
\newcommand{\Cg}{{\mathcal{g}}}
\newcommand{\Ch}{{\mathcal{h}}}
\newcommand{\Ci}{{\mathcal{i}}}
\newcommand{\Cj}{{\mathcal{j}}}
\newcommand{\Ck}{{\mathcal{k}}}
\newcommand{\Cl}{{\mathcal{l}}}
\newcommand{\Cm}{{\mathcal{m}}}
\newcommand{\Cn}{{\mathcal{n}}}
\newcommand{\Co}{{\mathcal{o}}}
\newcommand{\Cp}{{\mathcal{p}}}
\newcommand{\Cq}{{\mathcal{q}}}
\newcommand{\Cr}{{\mathcal{r}}}
\newcommand{\Cs}{{\mathcal{s}}}
\newcommand{\Ct}{{\mathcal{t}}}
\newcommand{\Cu}{{\mathcal{u}}}
\newcommand{\Cv}{{\mathcal{v}}}
\newcommand{\Cw}{{\mathcal{w}}}
\newcommand{\Cx}{{\mathcal{x}}}
\newcommand{\Cy}{{\mathcal{y}}}
\newcommand{\Cz}{{\mathcal{z}}}
\newcommand{\CA}{{\mathcal{A}}}
\newcommand{\CB}{{\mathcal{B}}}
\newcommand{\CC}{{\mathcal{C}}}
\renewcommand{\CD}{{\mathcal{D}}}
\newcommand{\CE}{{\mathcal{E}}}
\newcommand{\CF}{{\mathcal{F}}}
\newcommand{\CG}{{\mathcal{G}}}
\newcommand{\CH}{{\mathcal{H}}}
\newcommand{\CI}{{\mathcal{I}}}
\newcommand{\CJ}{{\mathcal{J}}}
\newcommand{\CK}{{\mathcal{K}}}
\newcommand{\CL}{{\mathcal{L}}}
\newcommand{\CM}{{\mathcal{M}}}
\newcommand{\CN}{{\mathcal{N}}}
\newcommand{\CO}{{\mathcal{O}}}
\newcommand{\CP}{{\mathcal{P}}}
\newcommand{\CQ}{{\mathcal{Q}}}
\newcommand{\CR}{{\mathcal{R}}}
\newcommand{\CS}{{\mathcal{S}}}
\newcommand{\CT}{{\mathcal{T}}}
\newcommand{\CU}{{\mathcal{U}}}
\newcommand{\CV}{{\mathcal{V}}}
\newcommand{\CW}{{\mathcal{W}}}
\newcommand{\CX}{{\mathcal{X}}}
\newcommand{\CY}{{\mathcal{Y}}}
\newcommand{\CZ}{{\mathcal{Z}}}
\newcommand{\Hom}{\mathop{\rm Hom}\nolimits}
\newcommand{\Homint}{\underline{\mathsf{Hom}}}
\newcommand{\Homsh}{\mathcal{H}om}
\newcommand{\RHom}{\mathop{\rm RHom}\nolimits}
\newcommand{\Ext}{\mathop{\rm Ext}\nolimits}
\newcommand{\Extsh}{\mathcal{E}xt}
\newcommand{\YExt}{\mathop{\rm YExt}\nolimits}
\newcommand{\Lim}{{\rm Lim}}
\newcommand{\Colim}{{\rm Colim}}
\DeclareMathOperator*{\Holim}{{\rm Holim}}
\DeclareMathOperator*{\Hocolim}{{\rm Hocolim}}
\newcommand{\ra}{\rightarrow}
\newcommand{\lra}{\longrightarrow}
\newcommand{\rap}{\stackrel{+}{\rightarrow}}
\newcommand{\Spec}{\mathop{{\bf Spec}}\nolimits}
\newcommand{\Spm}{\mathop{{\bf Spm}}\nolimits}
\newcommand{\Spf}{\mathop{{\bf Spf}}\nolimits}
\newcommand{\Proj}{\mathop{{\bf Proj}}\nolimits}
\newcommand{\Th}{\mathop{{\bf Th}}\nolimits}
\newcommand{\Sch}{\mathsf{Sch}}
\newcommand{\Sm}{\mathsf{Sm}}
\newcommand{\AnSm}{\mathsf{AnSm}}
\newcommand{\Ouv}{\mathsf{Ouv}}
\newcommand{\SmProj}{\mathsf{SmProj}}
\newcommand{\un}{\mathds{1}}
\newcommand{\Be}{{\scriptsize \mbox{\foreignlanguage{russian}{B}}}}
\newcommand{\Ob}{\mathsf{Ob}}
\newcommand{\Gm}{\mathbb{G}_\mathrm{m}}
\newcommand{\Ga}{\mathbb{G}_\mathrm{a}}
\newcommand{\pointille}{{_.}^.}
\newcommand{\card}{\mathop{\rm card}\nolimits}
\newcommand{\trace}{\mathop{\rm Tr}\nolimits}
\newcommand{\ord}{\mathop{\rm ord}\nolimits}
\newcommand{\charact}{\mathop{\rm char}\nolimits}
\newcommand{\rank}{\mathop{{\rm rank}}\nolimits}
\newcommand{\Gal}{\mathop{\rm Gal}\nolimits}
\newcommand{\SH}{\mathop{\mathbf{SH}}\nolimits}
\newcommand{\DM}{\mathop{\mathbf{DM}}\nolimits}
\newcommand{\DA}{\mathop{\mathbf{DA}}\nolimits}
\newcommand{\AnDA}{\mathop{\mathbf{AnDA}}\nolimits}
\newcommand{\Chow}{\mathop{\mathbf{Chow}}\nolimits}
\newcommand{\HI}{\mathop{\mathbf{HI}}\nolimits}
\newcommand{\MM}{\mathop{\mathbf{MM}}\nolimits}
\newcommand{\Cpl}{\mathop{\mathbf{Cpl}}\nolimits}
\newcommand{\Spt}{\mathop{\mathbf{Spt}}\nolimits}
\newcommand{\PSh}{\mathop{\mathbf{PSh}}\nolimits}
\newcommand{\Sh}{\mathop{\mathbf{Sh}}\nolimits}
\newcommand{\Mod}{\rm Mod}
\newcommand{\Mon}{\rm Mon}
\newcommand{\CMon}{\rm CMon}
\newcommand{\Cat}{\rm Cat}
\newcommand{\id}{{\rm id}}
\newcommand{\Ho}{\mathbf{Ho}}
\newcommand{\PreShv}{\mathbf{PSh}}
\newcommand{\Shv}{\mathbf{Sh}}
\newcommand{\D}{\mathsf{D}}
\newcommand{\Sym}{\mathsf{Sym}}
\newcommand{\Coh}{\mathsf{Coh}}
\newcommand{\Alt}{\mathsf{Alt}}
\newcommand{\SmCor}{\mathsf{SmCor}}
\newcommand{\Var}{\mathsf{Var}}
\newcommand{\An}{{\rm An}}
\newcommand{\Pic}{{\rm Pic}}
\newcommand{\sPic}{\mathcal{P}ic}
\newcommand{\NS}{{\rm NS}}
\newcommand{\Alb}{{\rm Alb}}
\newcommand{\Div}{{\rm Div}}
\newcommand{\Aut}{{\rm Aut}}
\newcommand{\Nis}{{\rm Nis}}
\newcommand{\Et}{{\rm Et}}
\newcommand{\Zar}{{\rm Zar}}
\newcommand{\Bor}{Bor}
\newcommand{\GL}{{\rm GL}}
\newcommand{\SL}{{\rm SL}}
\newcommand{\PGL}{{\rm PGL}}
\newcommand{\Gr}{{\rm Gr}}
\newcommand{\image}{\mathop{{\rm Im}}\nolimits}
\newcommand{\imm}{\mathop{{\rm im}}\nolimits}
\newcommand{\coimage}{\mathop{{\rm coim}}\nolimits}
\newcommand{\Lie}{\mathop{\rm Lie}\nolimits}
\newcommand{\End}{\mathop{\rm End}\nolimits}
\newcommand{\Isom}{\mathop{\rm Isom}\nolimits}
\newcommand{\Mor}{\mathop{\rm Mor}\nolimits}
\newcommand{\tildeExt}{\widetilde{\rm Ext}\mathstrut}
\newcommand{\UHom}{\mathop{\underline{\rm Hom}}\nolimits}
\newcommand{\UAut}{\mathop{\underline{\rm Aut}}\nolimits}
\newcommand{\Cent}{\mathop{\rm Cent}\nolimits}
\newcommand{\Norm}{\mathop{\rm Norm}\nolimits}
\newcommand{\Stab}{\mathop{\rm Stab}\nolimits}
\newcommand{\Quot}{\mathop{\rm Quot}\nolimits}
\newcommand{\Res}{\mathop{\rm Res}\nolimits}
\newcommand{\Ind}{\mathop{\rm Ind}\nolimits}
\newcommand{\Frac}{\mathop{\rm Frac}\nolimits}
\newcommand{\Id}{\mathop{\rm Id}\nolimits}
\newcommand{\CoInd}{\mathop{\rm CoInd}\nolimits}
\newcommand{\Tot}{\mathop{\rm Tot}\nolimits}
\newcommand{\DTot}{\mathop{\rm DTot}\nolimits}
\newcommand{\Pro}{\mathop{\rm Pro}\nolimits}
\newcommand{\Sus}{\mathop{\rm Sus}\nolimits}
\newcommand{\LSus}{\mathop{\rm LSus}\nolimits}
\newcommand{\Ev}{\mathop{\rm Ev}\nolimits}
\newcommand{\REv}{\mathop{\rm REv}\nolimits}
\newcommand{\Frob}{\mathop{{\rm Frob}}\nolimits}
\newcommand{\Tor}{\mathop{\rm Tor}\nolimits}
\newcommand{\Coker}{\mathop{\rm Coker}\nolimits}
\newcommand{\Ker}{\mathop{\rm Ker}\nolimits}
\newcommand{\supp}{\mathop{\rm Supp}\nolimits}
\newcommand{\Jac}{\mathop{\rm Jac}\nolimits}
\newcommand{\df}{\mathrm{df}}
\newcommand{\JG}{\mathcal{JG}}
\newcommand{\DG}{\mathcal{DG}}
\newcommand{\CCG}{\mathcal{CG}}
\newcommand{\PG}{\mathcal{PG}}
\newcommand{\sNS}{\mathcal{NS}}
\newcommand{\NSL}{\mathcal{NSL}}
\newcommand{\Picsm}{\mathcal{P}ic^\sm}
\newcommand{\Picsmc}{\mathcal{P}ic^\sm_*}
\newcommand{\adj}{\mathop{\rm adj}\nolimits}
\newcommand{\basechange}{\rm base\ change}
\newcommand{\Lotimes}{\mathbin{\stackrel{L}{\otimes}}}
\newcommand{\loccit}{[loc.$\;$cit.]}
\newcommand{\OFU}{\overline{\FU}}
\newcommand{\Ug}{\underline{g}}
\newcommand{\Un}{\underline{n}}
\newcommand{\Ur}{\underline{r}}
\newcommand{\Ux}{\underline{x}}
\newcommand{\diag}{\mathop{\rm diag}\nolimits}
\newcommand{\pr}{\mathop{\rm pr}\nolimits}
\newcommand{\Cone}{{\rm Cone}}
\newcommand{\CHo}{\mathop{\rm CH}\nolimits}
\newcommand{\LAlb}{\mathop{\rm LAlb}\nolimits}
\newcommand{\RPic}{\mathop{\rm RPic}\nolimits}
\newcommand{\Ab}{\mathop{\rm Ab}\nolimits}
\newcommand{\For}{\mathop{\rm For}\nolimits}
\newcommand{\Set}{\mathop{\rm Set}\nolimits}
\newcommand{\corexp}{\mathrm{cor}}
\newcommand{\et}{\mathrm{\'et}}
\newcommand{\eff}{\mathrm{eff}}
\newcommand{\qfh}{\mathrm{qfh}}
\newcommand{\gm}{\mathrm{gm}}
\newcommand{\op}{\mathrm{op}}
\newcommand{\aug}{\mathrm{aug}}
\newcommand{\coh}{\mathrm{coh}}
\newcommand{\homo}{\mathrm{hom}}
\newcommand{\dcoh}{\mathrm{dcoh}}
\newcommand{\red}{\mathrm{red}}
\newcommand{\sm}{\mathrm{sm}}
\newcommand{\ssm}{\mathrm{ssm}}
\newcommand{\gsm}{\mathrm{gsm}}
\newcommand{\ins}{\mathrm{ins}}
\newcommand{\ind}{\mathrm{ins}}
\newcommand{\nc}{\mathrm{nc}}
\newcommand{\mode}{\mathrm{mod}}
\newcommand{\sep}{\mathrm{sep}}
\newcommand{\s}{\mathrm{s}}
\newcommand{\nr}{\mathrm{nr}}
\newcommand{\tor}{{\rm tor}}
\newcommand{\opp}{{\rm opp}}
\newcommand{\steff}{{\rm st-eff}}
\newcommand{\Ex}{{\rm Ex}}
\newcommand{\tr}{{\rm tr}}
\newcommand{\perf}{{\rm perf}}
\newcommand{\fr}{{\rm fr}}
\newcommand{\gr}{{\rm gr}}
\newcommand{\str}{{\rm str}}
\newcommand{\ab}{{\rm ab}}
\newcommand{\num}{{\rm num}}
\newcommand{\pure}{{\rm pure}}
\newcommand{\an}{{\rm an}}
\newcommand{\psh}{{\rm psh}}
\newcommand{\adjo}{{\rm adj}}
\newcommand{\TODO}{{\color{red} TODO }}
\newcommand{\REF}{{\color{green} REF }}
		                        \usepackage{stmaryrd}


\newcommand\cosimparrowone{%
        \mathrel{\vcenter{\mathsurround0pt
                \ialign{##\crcr
                      \noalign{\nointerlineskip}$\rightarrow$\crcr
                      \noalign{\nointerlineskip}$\leftarrow$\crcr
                      \noalign{\nointerlineskip}$\rightarrow$\crcr
                }%
        }}%
    }
\newcommand\cosimparrowtwo{%
        \mathrel{\vcenter{\mathsurround0pt
                \ialign{##\crcr
                        \noalign{\nointerlineskip}$\rightarrow$\crcr
                        \noalign{\nointerlineskip}$\leftarrow$\crcr
                        \noalign{\nointerlineskip}$\rightarrow$\crcr
                        \noalign{\nointerlineskip}$\leftarrow$\crcr
                        \noalign{\nointerlineskip}$\rightarrow$\crcr

                }%
        }}%
}
    
\addbibresource{infcats.bib}
\date{\today}
\title{Categories and infinity-categories}
\hypersetup{
 pdfauthor={Simon Pepin Lehalleur},
 pdftitle={},
 pdfkeywords={},
 pdfsubject={},
 pdfcreator={Emacs 25.3.50.2 (Org mode 9.1.2)}, 
 pdflang={English}}
\begin{document}

\maketitle

\section*{Introduction}

Infinity category theory lies in the intersection of two major developments of 20th century mathematics: topology and category theory. Category theory is a very powerful framework to organize and unify mathematical theories. Infinity category theory extends this framework to settings where the morphisms between two objects form not a set but a topological space (or a related object like a chain complex). This situation arises naturally in homological algebra, algebraic topology and sheaf theory. 

This reading seminar will recall the foundational ideas of usual category theory and then make the transition to homotopical algebra and infinity categories. By the end of the seminar, the student will be familiar enough with infinity categories that they can navigate texts written in this new language.

\section*{Guidelines for the talks}

\begin{itemize}
\item The talks are given in English, should be given on the blackboard, and should last approximately 80 minutes to
allow for 10 minutes of questions.
\item Participants are expected to discuss their talk with me the week before they are scheduled to
speak (and bring with them a draft of their talk notes). The default appointment for this
discussion is Wednesday morning at 10am (the week before the talk) in room 108, Arnimallee 3 (if the participant
is not available at this time, they should email before this date to arrange a different time).
\item All required definitions and mathematical claims should be clearly stated; in particular, the
  definitions of all terms in italics in the descriptions below should be given.
\item The speaker should make sure that the assumptions and the claim are clear to the audience, in order for the
other participants to be able to follow proofs and explanations.
\end{itemize}

\section*{Program}

\subsection{October 18th: Introduction (Simon Pepin Lehalleur)}

Given by the lecturer.

\subsection{October 25th: Categories, Functors and natural transformations}

Following \cite[Chapter 1]{Riehl_context}, introduce the basic definitions of category theory.

\begin{itemize}
\item Skip the historical introduction before 1.1.
\item Present \S 1.1, with emphasis on some of the concrete examples of 1.1.3-4: Set, Top, Group, Ring, Mod$_R$, Man, Poset, Ch$_R$.
\item Present the three exercises of \S 1.1.
\item Explain categorical duality following \S 1.2.
\item Do some of the exercises of \S 1.2, according to your personal preference.
\item Explain the definition of a functor 1.3.1 and present some of the examples in 1.3.2: (i), (ii), (iii), (vi), (viii), (ix).
\item Present the rest of \S 1.3,skipping 1.3.3-4 and 1.3.15.
\item Explain the issue with duality of vector spaces at the beginning of \S 1.4.
\item Explain the definition of a natural transformation between functors 1.4.1; the rest of the section consists of examples, explain some in 1.4.3.
\item Give definitions 1.5.4, 1.5.7 and Theorem 1.5.9. Explain corollary 1.5.11.
\end{itemize}

\subsection{November 1st: Universal Properties, Representable Functors and the Yoneda Lemma}

Following \cite[Chapter 2]{Riehl_context}, discuss how category theory gives a powerful formalisation of the idea of ``universal properties''.

\begin{itemize}
\item Define representable functors and explain many examples as in \S 2.1. State the questions at the end of \S 2.1 as motivation for the Yoneda lemma.
\item State and prove the Yoneda lemma 2.2.4.
\item Explain corollary 2.2.8.
\item Present \S 2.3. The example 2.3.7 of tensor products, seen through the perspective of representable functors, is a very good one.
\item Present the definition 2.4.1-2 and explain their relation with slice categories as in 2.4.6.
\end{itemize}

\subsection{November 11th: Limits and Colimits}

Following \cite[Chapter 3]{Riehl_context}, discuss the fundamental notion of limits and colimits of functors.

\begin{itemize}
\item Explain all of \S 3.1, perhaps skipping a few examples, but essentially everything here is important!
\item Discuss how to construct all limits in the category of sets explicitely: 3.2.6 and 3.2.13.
\item Explain what it means for a functor to preserve (co)limits 3.3.1, skip the rest of \S 3.3.
\item Explain 3.4.2. State 3.4.6 and its dual 3.4.11 (without the reflection part). State Theorem 3.4.12.
\item Give the definition of complete and cocomplete categories, and some examples and counter-examples as in the beginning of \S 3.5.
\end{itemize}

\subsection{November 15th: Adjoint Functors}

Following \cite[Chapter 4]{Riehl_context}, discuss the adjoint functors and its relationship to all the category theory we know so far.

\begin{itemize}
\item Present the definition of adjoint functors and some of the examples of \S 4.1.
\item Explain units, counits and the alternative formulation of adjunction in \S 4.2.
\item Skip \S 4.3.
\item Present all the results in \S 4.4 and sketch at least one proof.
\item Explain the fundamental results 4.5.1-3 and its applications 4.5.7-11. Explain Lemma 4.5.13.
\end{itemize}

\subsection{November 22th: Simplicial sets}

A \emph{simplicial set} is an abstract combinatorial object, which allows to model topological space up to homotopy. We have a set $K_n$ for every $0 \leq n < \infty$, which we can think of as maps from an $n$-dimensional triangle into a space, and various morphisms $\delta_i: K_n \to K_{n-1}, \sigma_i: K_{n-1} \to K_n$ telling us how the triangles fit together.

\begin{itemize}
\item Define simplicial objects in a category as in \cite[8.1.4]{Weibel} and prove \cite[8.1.1-3]{Weibel}. Explain example \cite[8.1.5]{Weibel}. Define morphisms of simplicial sets as natural transformations between functors and explain that simplicial sets form a category $\mathrm{sSet}$.
\item Define simplicial spheres and simplicial horns as in \cite[5.1-3]{RiehlSS}.
\item Define the nerve $N(C)$ of a category $C$ \cite[Example 1.2]{Groth}. Define the functor $N:\mathrm{Cat}\ra \mathrm{sSet}$ and prove that it is fully faithful. Discuss the special case of the nerve of a partially ordered set, seen as a category.%
\item Define the geometric realisation of a simplicial set \cite[8.1.6]{Weibel}. Draw pictures of the geometric realisations of many of the examples considered so far.
\item Define the product of two simplicial sets $K_{\bullet}$ and $K'_{\bullet}$ by $(K\times K')_{n}=K_{n}\times K'_{n}$ and the obvious morphisms. State that it is the categorical product of $K_{\bullet}$ and $K'_{\bullet}$ in $\mathrm{sSet}$.
\item Given two simplicial sets $K$ and $K'$, define the mapping space $\mathrm{Map}(K,K')$, which is also a simplicial set, by $\mathrm{Map}(K,K')_{n}=\mathrm{Hom}(\Delta^{n}\times K,K')$. Prove that, for a fixed simplicial set $K$, the functor $\mathrm{Map}(K,-):\mathrm{sSet}\rightarrow \mathrm{sSet}$ is a right adjoint of the functor $-\times K:\mathrm{sSet}\rightarrow \mathrm{sSet}$.
\item Explain how a simplicial object in an abelian category gives rise to a chain complex \cite[8.2.1]{Weibel} (actually check $d^{2}=0$). Define simplicial homology \cite[8.2.3]{Weibel}, and explain the relationship between the singular homology of a topological space and the simplicial homology of its singular complex \cite[8.2.4]{Weibel}.
\end{itemize}
  
\subsection{November 29th: $\infty$-categories}

Infinity-categories, at least in the context of this seminar, will always be what is ofter called a \emph{quasi-category}. A quasi-category is a special type of simplicial set, satisfying a lifting condition similar but weaker that the one of Kan simplicial sets discussed in the previous talk. We will see that one can in particular associate a quasi-category to any category, in a way that does not lose information. Quasi-categories are much more flexible, though: informally, they allow to have morphisms \emph{between} morphisms (called $2$-morphisms, and represented by $2$-simplices), $3$-morphisms between $2$-morphisms, and so on.

\begin{itemize}
\item Define the singular simplicial set of a topological space \cite[8.2.4]{Weibel}. Prove that the geometric realisation is a left adjoint functor of the singular simplicial set functor.
\item Define Kan simplicial sets \cite[8.2.7]{Weibel} and Kan fibrations \cite[Ex.2.0.0.1]{HTT}.
\item Explain the definition of simplicial homotopy groups of fibrant simplicial sets \cite[8.3.1]{Weibel}. State without proof that, for a Kan simplicial set $X_{\bullet}$, we have for all $n\geq 0$ and every $x_{0}\in X_{0}$ that $\pi_{n}(X_{\bullet},x_{0})\simeq \pi_{n}(|X_{\bullet}|,x_{0})$.
\item Define normalized complexes of a simplicial object in an abelian category and state the Dold-Kan correspondence \cite[8.3.6-8]{Weibel}. 
\item Define an $\infty$-category \cite[Def.1.7]{Groth}, \cite[Def.1.1.2.4]{HTT}. Explain that, by definition, we have a (usual) category of $\infty$-categories which is a full subcategory of $\mathrm{sSet}$.
\item Show that for any category, its nerve is an ∞-category. Give an example of a category whose nerve is not a Kan complex. State and prove \cite[Proposition 1.1.2.2]{HTT} which caracterize those infinity categories which come from categories.
\item Define the simplicial set of functors between two ∞-categories \cite[Def.2.1]{Groth}, \cite[Not.1.2.7.2]{HTT}. State that it is an ∞-category \cite[Prop.2.5(i)]{Groth}, \cite[1.2.7.3]{HTT}.
\end{itemize}

\subsection{December 6th: join and slices }  

The \emph{join} of two topological spaces is the topological space we get by joining every point in one to every point in the other. This lecture shows how to define this for simplicial sets. A special case is when one space is a single point. This is called the \emph{cone} for obvious reasons. Joins are needed for the definition of \emph{slice} categories, which are needed for the definition of \emph{limits} in infinity-categories. The ``slice'' of a morphism of topological spaces $f: X \to Y$ is something like the space of pairs $(x, \gamma)$ where $x \in X$ is a point and $\gamma: [0, 1] \to Y$ is a path starting from $f(x)$.

This lecture will cover the following: 

\begin{itemize}
\item Define the cone of a topological space, and draw the picture \cite[pp.8-9]{Hat}. %
\item Define the join of two topological spaces, and draw the picture \cite[p.9]{Hat}. %
\item Define the join of two simplicial sets \cite[Def.2.11]{Groth}, \cite[Def.1.2.8.1]{HTT}. %
\item Show that there are isomorphisms $∆^{i} {\star} ∆^{j} \cong ∆^{i+j+1}$. %
\item Define the right cone and left cone of a simplicial set, and describe them explicitly \cite[Ex.2.14]{Groth}, \cite[Not.1.2.8.4]{HTT}. %
\item \emph{Prove} that for any two ∞-categories $S, S'$, the join $S \star S'$ is an ∞-category \cite[Prop.1.2.8.3]{HTT}. %
\item Define the overcategory  $C_{/p}$ of a map $p$ by its universal property \cite[Prop.2.17]{Groth}, \cite[Prop.1.2.9.2]{HTT}. %
\item Define $C_{/p}$  explicitly \cite[Proof of Prop.1.2.9.2]{HTT}. %
\item State (without proof) that $C_{/p}$ is an ∞-category \cite[Prop.1.2.9.3]{HTT}. %
\item Define the undercategory by a universal property, and explicitly \cite[Rem.1.2.9.5]{HTT}. %
\item Given a morphism of topological spaces $p: Y \to X$, explicitly describe the ∞-category $Sing_\bullet(X)_{/Sing(p)}$. %
\end{itemize}  

\subsection{December 13th: Limits and Colimits in infinity categories}

There is a theory of limits and colimits in infinity categories which generalizes the one for usual categories. This talk looks at the basic definitions, and also at examples in the infinity-category of topological spaces, which correspond to classical constructions of \emph{homotopy limits and colimits}. The idea is that classical limits and colimits in topological spaces are not homotopy-invariant, and homotopy limits and colimits correct this.

\begin{itemize}
\item Define the ∞-category of topological spaces. The $n$-simplices consist of:
\begin{enumerate}
 \item A set of $n{+}1$ topological spaces $X_0, \dots, X_n$.

 \item For each $i = 0, \dots, n{-}1$ and $a = 1, \dots, n{-}i$, a morphism $h_{i,i{+}a}: X_i {\times} \square^{a{-}1}_{top} \to X_{a{+}i}$.% (by convention, $\square^{0}_{top} = \{\ast\}$ and $\square^{-1}_{top} = \varnothing$).

 \item The morphisms $h_{i,j}$ are required to satisfy the compatibility condition: For every $a, b$, the restriction of $h_{i,k{+}a{+}b}$ to $X_i{\times}\square^{a{-}1}_{top} {\times} \square^{b{-}1}_{top} \subseteq X_{i}{\times}\square^{a{+}b{-}1}_{top}$ is the composition $h_{i{+}a,i{+}a{+}b} \circ (h_{i,i{+}a} \times \id_{\square^{b{-}1}_{top}})$. Here, the inclusion $\square^{a{-}1}_{top} {\times} \square^{b{-}1}_{top} \subseteq \square^{a{+}b{-}1}_{top}$ is given by $((t_1, \dots, t_{a-1}), (s_1, \dots, s_{b-1})) \mapsto (t_1, \dots, t_{a-1}, 0, s_1, \dots, s_{b-1})$.
\end{enumerate}
\item Define initial and final objects \cite[§2.4]{Groth}, \cite[§1.2.12.3]{HTT}. %
\item Show that the ∞-category of a partially ordered set has an initial (resp. final) object if and only if it has a minimal (resp. maximal) element.
\item Show that the one point topological space is a final object in the ∞-category of topological spaces. %
  State that a topological space homotopy equivalent to a one point topological space is a final object. %
\item Define colimits and limits \cite[Def.1.2.13.4]{HTT}. %
\item Observe that initial (resp. final) objects are colimits (resp. limits) of the empty diagram. %
\item Show that limits / colimits in (nerves of) partially ordered sets are infimums / supremums. %
\item Show that limits / colimits in (nerves of) categories are usual limits / colimits.
\item Define pushout and pullback squares \cite[Def.2.29]{HTT}. %   \\ 
\item Define the \emph{homotopy pushout} of a diagram $Z \stackrel{f}{\leftarrow} X \stackrel{g}{\to} Y$  of topological spaces as $ \frac{Y \amalg [0,1]\times X \amalg Z}{(f(X) \sim \{0\} \times X, \ g(X) \sim \{1\} \times X)}$. %
\item Claim that this gives a pushout square in the ∞-category of topological spaces. (Optional) Show this claim. Explain that homotopy pushouts are homotopy invariant, in the sense that, for a map of diagrams as above whose components are weak equivalences, the resulting map is a weak equivalence.%
\item Define the \emph{homotopy pullback} of a diagram $Z \stackrel{f}{\rightarrow} X \stackrel{g}{\leftarrow} Y$  of topological spaces as $\{( z, \gamma, y) \in Z \times \hom([0,1], X) \times Y : \gamma(0) = f(z), \gamma(q) = g(y) \}$. Claim that this gives a pullback square in the ∞-category of topological spaces. %
\end{itemize}
  
\subsection{December 20th: Homotopical algebra and Model Categories}

\subsection{January 10th: The Quillen and Joyal model structures on Simplicial Sets}

\subsection{January 17th: Straightening and unstraightening, the Yoneda lemma for infinity categories
}

\subsection{January 24th: Adjoints in infinity categories, limits and colimits revisited
}

\subsection{January 31th: Stable infinity-categories
}

\subsection{February 7th: Monoidal infinity-categories}

\subsection{February 14th: Infinity topoi}

\printbibliography

\end{document}
\documentclass[a4paper]{amsart}


% Standard Packages
\usepackage{amssymb}
%\usepackage{amscd}
%\usepackage{enumitem}
\usepackage{hyperref}
\usepackage[utf8]{inputenc}
\usepackage{newunicodechar}
%\usepackage{varioref}
\usepackage[arrow,curve,matrix]{xy}
\usepackage{mdframed}

%% Graphics Packages
%\usepackage{colortbl}
%\usepackage{graphicx}
%\usepackage{tikz}

% Font packages
\usepackage{mathrsfs}

\DeclareUnicodeCharacter{00A0}{ }


%
% GENERAL TYPESETTING
%
%
%% Colours for hyperlinks
%\definecolor{linkred}{rgb}{0.7,0.2,0.2}
%\definecolor{linkblue}{rgb}{0,0.2,0.6}

% Limit table of contents to section titles
\setcounter{tocdepth}{1}

% Numbering of figures (see below for numbering of equations)
\numberwithin{figure}{section}

% Add an uparrow to the bibliography entries, just before the back-list of references
\usepackage[hyperpageref]{backref}
\renewcommand{\backref}[1]{$\uparrow$~#1}

% Numbering of parts in roman numbers
%\renewcommand\thepart{\rm \Roman{part}}

% Sloppy formatting -- often looks better
\sloppy

% Changes the layout of descriptions and itemized lists. The indent specified in
% the original amsart style is too much for my taste.
%\setdescription{labelindent=\parindent, leftmargin=2\parindent}
%\setitemize[1]{labelindent=\parindent, leftmargin=2\parindent}
%\setenumerate[1]{labelindent=0cm, leftmargin=*, widest=iiii}



%
% Input characters
%
\newunicodechar{α}{\ensuremath{\alpha}}
\newunicodechar{β}{\ensuremath{\beta}}
\newunicodechar{χ}{\ensuremath{\chi}}
\newunicodechar{δ}{\ensuremath{\delta}}
\newunicodechar{∆}{\ensuremath{\Delta}}
\newunicodechar{η}{\ensuremath{\eta}}
\newunicodechar{γ}{\ensuremath{\gamma}}
\newunicodechar{Γ}{\ensuremath{\Gamma}}
\newunicodechar{ι}{\ensuremath{\iota}}
\newunicodechar{κ}{\ensuremath{\kappa}}
\newunicodechar{λ}{\ensuremath{\lambda}}
\newunicodechar{Λ}{\ensuremath{\Lambda}}
\newunicodechar{μ}{\ensuremath{\mu}}
\newunicodechar{ω}{\ensuremath{\omega}}
\newunicodechar{Ω}{\ensuremath{\Omega}}
\newunicodechar{π}{\ensuremath{\pi}}
\newunicodechar{φ}{\ensuremath{\phi}}
\newunicodechar{Φ}{\ensuremath{\Phi}}
\newunicodechar{ψ}{\ensuremath{\psi}}
\newunicodechar{Ψ}{\ensuremath{\Psi}}
\newunicodechar{ρ}{\ensuremath{\rho}}
\newunicodechar{σ}{\ensuremath{\sigma}}
\newunicodechar{Σ}{\ensuremath{\Sigma}}
\newunicodechar{τ}{\ensuremath{\tau}}
\newunicodechar{θ}{\ensuremath{\theta}}
\newunicodechar{Θ}{\ensuremath{\Theta}}


\newunicodechar{∞}{\ensuremath{\infty}}
\newunicodechar{→}{\ensuremath{\to}}
\newunicodechar{⨯}{\ensuremath{\times}}
\newunicodechar{∪}{\ensuremath{\cup}}
\newunicodechar{∩}{\ensuremath{\cap}}
\newunicodechar{⊇}{\ensuremath{\supseteq}}
\newunicodechar{⊃}{\ensuremath{\supset}}
\newunicodechar{⊆}{\ensuremath{\subseteq}}
\newunicodechar{⊂}{\ensuremath{\subset}}
\newunicodechar{≥}{\ensuremath{\geq}}
\newunicodechar{≤}{\ensuremath{\leq}}
\newunicodechar{∈}{\ensuremath{\in}}
\newunicodechar{◦}{\ensuremath{\circ}}
\newunicodechar{°}{\ensuremath{^\circ}}
\newunicodechar{…}{\ifmmode\mathellipsis\else\textellipsis\fi}
\newunicodechar{⊗}{\ensuremath{\otimes}}

%operators
\newcommand{\salt}{\mathrm{s\mbox{-}alt}}
\newcommand{\stimes}{⨯^{\salt}}
\newcommand{\isom}{\cong}
\newcommand{\tensor}{\otimes}
\newcommand{\Frac}{\mathsf{Frac}} % Quotient field
\newcommand{\Gal}{\mathsf{Gal}} % Galois group
\newcommand{\Aut}{\mathsf{Aut}} % Automorphism group


%theorem environments
\theoremstyle{theorem}
\newtheorem{thm}{Theorem}
\newtheorem*{thmU}{Theorem}
\newtheorem{theo}{Theorem}
\newtheorem{coro}[thm]{Corollary}
\newtheorem{cor}[thm]{Corollary}
\newtheorem*{corU}{Corollary}
\newtheorem{lemm}[thm]{Lemma}
\newtheorem{lemma}[thm]{Lemma}
\newtheorem{propo}[thm]{Proposition}
\newtheorem{prop}[thm]{Proposition}
\newtheorem{conj}[thm]{Conjecture}


\theoremstyle{definition}
\newtheorem{defi}[thm]{Definition}
\newtheorem{defn}[thm]{Definition}
\newtheorem{propdef}[thm]{Definition and Proposition}
\newtheorem{obse}[thm]{Observation}
\newtheorem{rema}[thm]{Remark}
\newtheorem{rem}[thm]{Remark}
\newtheorem{remi}[thm]{Reminder}
\newtheorem{exam}[thm]{Example}
\newtheorem{summ}[thm]{Summary}
\newtheorem{nota}[thm]{Notation}
\newtheorem{warn}[thm]{Warning}
\newtheorem*{ques}{Question}
\newtheorem*{exer}{Exercise}




%letters
\DeclareSymbolFontAlphabet{\scr}{rsfs}
\newcommand{\Fh}{\mathcal{F}}
\newcommand{\Oh}{\mathcal{O}}
\newcommand{\OO}{\mathcal{O}}

\newcommand{\CC}{\mathbb{C}}
\newcommand{\EE}{\mathbb{E}}
\newcommand{\FF}{\mathbb{F}}
\newcommand{\Fp}{\mathbb{F}\!{_p}}
\newcommand{\cG}{\mathcal{G}}
\newcommand{\LL}{\mathbb{L}}
\newcommand{\NN}{\mathbb{N}}
\newcommand{\PP}{\mathbb{P}}
\newcommand{\QQ}{\mathbb{Q}}
\newcommand{\RR}{\mathbb{R}}
\newcommand{\Z}{\mathbb{Z}}
\newcommand{\ZZ}{\mathbb{Z}}
\renewcommand{\AA}{\mathbb{A}}
\newcommand{\Q}{\mathbb{Q}}
\newcommand{\F}{\mathbb{F}}
\newcommand{\Pe}{\mathbb{P}}
\newcommand{\m}{\mathfrak{m}}
\newcommand{\p}{\mathfrak{p}}
\newcommand{\q}{\mathfrak{q}}

\newcommand{\Top}{{Top}}
%\newcommand{\Top}{\mathsf{Top}}
\newcommand{\pTop}{{Top}_\ast}
%\newcommand{\pTop}{\mathsf{Top}_\ast}
\newcommand{\grVec}{{Vec}}
%\newcommand{\grVec}{\mathsf{Vec}}
\newcommand{\sSet}{{\mathsf{Set}_\Delta}}
\newcommand{\Spaces}{{\mathcal{S}}}
\newcommand{\PreShv}{{{P}}}
%\newcommand{\PreShv}{{\mathcal{P}}}
\newcommand{\Shv}{{{Shv}}}
%\newcommand{\Shv}{{\mathsf{Shv}}}
\newcommand{\Op}{{{Op}}}
%\newcommand{\Op}{{\mathsf{Op}}}
\newcommand{\Sp}{{{Sp}}}
%\newcommand{\Sp}{{\mathsf{Sp}}}

%Special
\newcommand{\val}{\mathrm{val}}
\newcommand{\shval}{\mathrm{shval}}
\DeclareMathOperator{\id}{id}
\DeclareMathOperator{\rank}{rank}
\DeclareMathOperator{\trd}{tr{.}d}
\DeclareMathOperator{\uhom}{\underline{hom}}
\DeclareMathOperator{\Fun}{Fun}
\DeclareMathOperator{\Map}{Map}
%\DeclareMathOperator{\char}{char}

\title{$∞$-categories seminar \\
Talk III. $∞$-Categories}

\renewcommand{\thesection}{\Roman{section}}
\setlength{\parskip}{1em}
\setcounter{tocdepth}{2}

\begin{document}

\begin{exer}
Let $p: X → S$ be a morphism of simplicial sets, and $f: ∆^1 → X$ an edge, with source $x: ∆^0 → X$ (that is, $x$ is the composition of $f$ with the canonical inclusion $∆^0 ⊆ ∆^1$). Show that the following are equivalent.
\begin{enumerate}
 \item[(A)] The induced map $X_{ / f} → X_x \times_{S_{/{p(x)}}} S_{/ p(f)}$ is a trivial Kan fibration.

 \item[(B)] For every $n ≥ 2$, and every commutative diagram
 \[ \xymatrix{
 ∆^1 \ar[d] \ar[dr]^f & \\
 \Lambda^n_0 \ar[r] \ar[d] & X \ar[d]^p \\
 ∆^n \ar[r] \ar@{-->}[ur] & S
 } \] 
there is a dashed morphism making the diagram commutative.
 \end{enumerate}
\end{exer}

\emph{Step a).} Recall that a morphism is a trivial Kan fibration if and only if it has the left lifting property with respect to the morphisms $\partial ∆^n → ∆^n$ for all $n ≥ 0$. So (A) is equivalent to:
\begin{enumerate}
 \item[(A')] For all $n ≥ 0$ and every commutative diagram
 \begin{equation} \label{A}
  \xymatrix{
\partial ∆^n \ar[d] \ar[r] & X_{ / f} \ar[d] \\
∆^n \ar[r] \ar@{-->}[ur] & X_x \times_{S_{/{p(x)}}} S_{ / p(f)}
 } 
 \end{equation}
there is a dashed morphism making the diagram commutative.
\end{enumerate}

\emph{Step b).} Recall that by the universal property of overcategories, the upper horizontal morphism in \eqref{A} corresponds to a unique commutative diagram
 \begin{equation} \label{B}
 \xymatrix{
∆^1 \ar@/^3ex/[rr]^f \ar[r] & ∆^1 * \partial ∆^n \ar[r] & X
} 
\end{equation}
where $∆^1 ⊆∆^1 * \partial ∆^n$ is the canonical inclusion (and conversely, such a diagram corresponds to a unique morphism $\partial ∆^n → X_{ / f}$). Moreover, by the universal property of fibre products, the lower horizontal morphism in \eqref{A} corresponds to a unique commutative diagram
\[ \xymatrix{
∆^n \ar[r] \ar[d] & S_{ / p(f)} \ar[d] \\
X_x  \ar[r] & {S_{/{p(x)}}}
} \]
(where the lower morphism and right morphism are the canonical ones), and by the universal property of overcategories, such a diagram corresponds uniquely to a commutative diagram 
 \begin{equation} \label{C}
 \xymatrix{
∆^0 \ar@/^3ex/[rr]^x \ar[r] & ∆^{0} * ∆^n \ar[r] \ar[d] & X \ar[d]^p \\
∆^1 \ar@/_3ex/[rr]_{p(f)} \ar[r] &∆^1 * ∆^n \ar[r] & S 
} 
\end{equation}
where $∆^0 ⊆ ∆^0 {*} \partial ∆^n$, $∆^1 ⊆∆^1 {*} \partial ∆^n$, $∆^{0} {*} ∆^n ⊆ ∆^{1} {*} ∆^n$ are the canonical ones.

\emph{Step c).} Observe that the commutivity of \eqref{A} (without the dashed morphism) is equivalent to the requirement that \eqref{B} and \eqref{C} fit together into a commutative diagram
\begin{equation} \label{D}
\xymatrix{
∆^1 \ar[dr]^f \ar[d] & \\
(∆^1 * \partial ∆^n) \cup (∆^0 * ∆^n) \ar[r] \ar[d] & X \ar[d]^p \\
∆^1 * ∆^n \ar[r] & S
}
\end{equation}

\emph{Step d).} Observe that the inclusion
\[ (∆^1 * \partial ∆^n) \cup (∆^0 * ∆^n) ⊆ ∆^1 * ∆^n \]
is canonically isomorphic to the inclusion
\[ \Lambda^{n+2}_0 ⊆ ∆^{n+2}. \]
(Indeed, $(∆^0 * ∆^n) ⊆ ∆^1 * ∆^n$ is the inclusion of the face $d_1 ∆^{n+2} ⊆ ∆^{n+2}$. Moreover, $\partial ∆^n$ is the union of the faces $d_i∆^n ⊆ ∆^n$ for $0 ≤ i ≤ n$, and $∆^1 * d_i∆^n ⊆ ∆^{n+2}$ is the inclusion of the face $d_{i+2}∆^{n+2} ⊆ ∆^{n+2}$). Conclude that \eqref{D} is isomorphic to
\begin{equation} \label{E}
\xymatrix{
∆^1 \ar[dr]^f \ar[d] & \\
\Lambda^{n+2}_0  \ar[r] \ar[d] & X \ar[d]^p \\
∆^{n+2} \ar[r] & S
}
\end{equation}


%Recall that there is a canonical isomorphism $∆^1 * ∆^n \cong ∆^{n+2}$. Moreover, under this isomorphism, $∆^{\{1\}} * ∆^n$ corresponds to $∆^{\{1, \dots, n+2\}}$, and $∆^1 * \partial ∆^n$ corresponds to the nerve of the partially ordered set of subsets $I ⊆ \{0, \dots, n+2\}$ such that $\{2, 3, \dots, n+2\} \not\subseteq I$.

%\emph{Step d).} Observe that since $p: X \to S$ is an inner fibration, and for all $i ≥ 2$ there are inclusions 
%\[ ∆^1 ⊆ \Lambda^{\{0,1,2, \dots, i{-}1,i{+}1, \dots, {n+2}\}}_{1} ⊆ (∆^1 * \partial ∆^{\{2, \dots, n{+}2\}}) \cup (∆^0 * ∆^{\{2, \dots, n{+}2\}}) ⊆ ∆^{n+2} \] 
%compatible with the canonical isomorphism $∆^1 * ∆^n \cong ∆^{n+2}$, every diagram \eqref{D} can be completed to a diagram of the form
%\begin{equation} \label{E}
%\xymatrix{
%∆^1 \ar[dr]^f \ar[d] & \\
%(∆^1 * \partial ∆^n) \cup (∆^0 * ∆^n) \ar[r] \ar[d] & X \ar[d]^p \\
%∆^1 * ∆^n \ar[r] & S
%}
%\end{equation}

\emph{Step e).} Conclude that \eqref{A} admits a dashed morphism making the diagram commute if and only if the corresponding diagram 
\begin{equation} \label{F}
\xymatrix{
∆^1 \ar[dr]^f \ar[d] & \\
\Lambda^{n+2}_0  \ar[r] \ar[d] & X \ar[d]^p \\
∆^{n+2} \ar[r] \ar@{-->}[ur] & S
}
\end{equation}
admits a dashed morphism making the diagram commute.















\end{document}


